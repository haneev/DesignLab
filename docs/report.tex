\documentclass[12pt]{article}

\usepackage{graphicx}

\title{Report of FedWeb DesignLab Challenge}
\author{
        Han van der Veen \\ s1007130
            \and
			Rik van Outersterp \\ s1010875
}
\date{\today}

\begin{document}
\maketitle

%\begin{abstract}
%Aangezien het een verslag is en geen paper lijkt mij een abstract niet nodig.
%\end{abstract}

\section{Introduction}
This document elaborates our submission for the \textit{Design Challenge} assignment of the course \textit{Information Retrieval} at the University of Twente. 
For this assignment a web result page for aggregated web search had to designed.
The results of this web search consist of the results of multiple (independent) search engines.
The designed result page is planned to be used by the course' supervisors in a successive user experiment of which the purpose is to determine which pages give the best overall user experience, followed by a presentation of this study at the Text Retrieval Conference (TREC) in November 2014.

In this document we elaborate our web result pages.
The remainder of the document is setup as follows.
In section~\ref{sec:designs} our two designs are presented.
A proposal for evaluating our designs is presented in section~\ref{sec:evaluation}.
This document is concluded in section~\ref{sec:conclusion}.

\section{Designs}
\label{sec:designs}
Both our designs are build upon the same system.
The ranking of the search results is in both designs based on the title of the result.
A higher rank is achieved by a search result when the title contains the search term.
An even higher rank can be achieved when the title starts with the search term or even matches the search term exactly (note: case insensitive).

Since the assignment requires to have as many designs as group members, we have developed two designs.
We decided to put our own personal ideas in a design, such that two different designs were developed instead of two possibly very similar designs.
In the following subsections both designs are presented.

% Han's layout
\subsection{Design One}
\label{sec:layoutHan}
For the first design (Design One, see figure~\ref{fig:designOne}) we have chosen to blend in all the search engines.
By blending it together, all the results get the same amount of space in the layout and should be therefore considered as equally important. 
This allows the user to be able to determine the best result by himself instead of letting the system decide what the best result should be.
As mentioned this design uses the sorting system that was described in the beginning of section~\ref{sec:designs}.

\begin{figure}[h!]
  \centering
    \includegraphics[width=1.0\textwidth]{designOne.png}
  \caption{Screenshot of Design One.}
\label{fig:designOne}
\end{figure}

Each block in the design represents a search result with its title, description, and a link to the original source. 
Each block has the same width and height to give each result the same importance. 
Furthermore the top left result is the first result and can be seen as the most relevant result to the query.
As can be seen from the result reports the relevance is ordered from left to right, and then from top to bottom. 
The source is also shown in the block of each result.
This allows the user to scan the results for results of specific search engines. 

In our design the whole screen is used, such that there is no unused space. 
The blocks will also float to other screen sizes, such that each browser has the same, correct display of the result page.
The dynamic setup of the design allows us to show as much information to the user as possible. 
However, it should be noted that too much information can distract the user, making searching for a desired result inefficient.
By giving each search result the same width and height, and thus a structured result page, we aim to prevent this. 

% Rik's layout
\subsection{Design Two}
\label{sec:layoutRik}
The second design (Design Two, see figure~\ref{fig:designTwo}) has an other approach than our first design.
In this design the results are ordered by category first before ordering by relevance.
In order to do this each search engine had to be allocated to a category.
For this we used the file \textit{FW14-engines.txt}, which consists of all search engines used to generate the FedWeb result data.
In total 21 categories were used to group similar search engines, and thus there search results, together.
In appendix~\ref{app:enginetypes} the allocation of the search engines to categories is shown.

\begin{figure}[h!]
  \centering
    \includegraphics[width=1.0\textwidth]{designTwo.png}
  \caption{Screenshot of Design Two.}
\label{fig:designTwo}
\end{figure}

The design is built while keeping in mind the current use of widescreens.
In our opinion long vertical lists may not be the most userfriendly for widescreens, thus we looked into a design that has a more horizontal approach.

The design consists of an undefined number of rows with each consisting of a maximum of five categories.
The categories are sorted based on the retrieval of the results, so the category of the first retrieval will be the category in the upper left of the display, the next new category will be next to it on the right, et cetera.
Each category takes up 20\% of the total screen width.
This is chosen to maximize the available space on screens instead of waisting space because it is considered more modern.

Each row has a maximum height, such that the user can see already more categories if there are more than five.
It should make the user immediately aware of the fact that there are more types of search results, instead of having to scroll down first.

We decided to not develop a responsive layout due to time constraints.
Therefore it is important to note that this design was developed on a computer system with a screen resolution of 1680 x 1050 pixels.
When using this resolution, or a higher resolution like 1920 x 1080 pixels, the display of the design should be correct.
When considering the screen width, all resolutions from 1024 x 768 pixels and higher should be able to display five categories beside each other, although the text might be not userfriendly to read at the lower resolutions.
When considering the screen height, all resolutions from 1280 x 1024 pixels and higher should be able to display the headings of the second row of categories.

\section{How to evaluate this project designs}
\label{sec:evaluation}
To test new designs at least two designs to compare have to be established. 
In order to get a good comparison the amount of variables of a design has to be minimized. 
In general the design that answers a user's question as best is considered as the best design.

To perform such an evaluation we need to have a control variable, which can be in our case ten results for each search engine in a tabbed environment. The user can find the information by clicking the correct tab. The other designs we have are blended designs.

Two aspects will be evaluated: are the results the correct results, and how fast will the user find his answer?

To find out if a design performs better than the baseline design, we will measure the time it takes for the user to have his desired answer. The faster the user finds the correct result, the better a design performs. The time is measured from the moment the user enters the query until clicking on the correct result. 

In order to perform this evaluation several test subjects are needed that will perform several queries on the different designs. They may not execute the same query in two designs, because that will give them some notion of what they are looking for. For example, every person has to perform thirty queries: ten on the base design, ten on the first design, and ten on the second design. The time is compared for the same query for different persons.

To automate this process our blended designs are evaluated with a list of correct results (answers) that is given for each query by experts as described by~\cite{lalmas2011aggregated}. Our automated blended designs must have those answers as a result, or else our design is wrong. 
The more answers we have in our result pages, the better our design is performing.
This can be determined automatically by simple testing if the answers are in the results or not.

A design is considered better than an other design when it outscores the other design in both tests.

\section{Conclusions}
\label{sec:conclusion}
We developed two different designs for a web result page.
Design One presents the search results in equally sized blocks, ordered only by relevance.
Design Two on the other hand orders by category first, followed by an order by relevance within each category.
Since both designs are mainly based on their type of ordering, it seems clear that the way of ordering has a large influence on a design.
However, if one or both of our designs are `good' designs cannot be concluded at this moment.
To determine this our designs should be compared to other designs of result pages, for example by an execution of our proposed evaluation.

%\section{Checklist}
%\begin{enumerate}
%\item Check if our readme is valid with a new clean install
%\end{enumerate}

\bibliographystyle{abbrv}
\bibliography{libfile}

\appendix
\section{Search Engine Categories}

\textbf{Blogs}
\begin{itemize}
\item	Google Blogs	
\item	LinkedIn Blog	
\item	Tumblr	
\item	WordPress	
\end{itemize}
\textbf{Books}
\begin{itemize}
\item	Goodreads	
\item	Google Books	
\item	NCSU Library 	
\item	Wikibooks	
\end{itemize}
\textbf{Career}
\begin{itemize}
\item	Glassdoor	
\item	Jobsite	
\item	LinkedIn Jobs	
\item	Simply Hired	
\item	USAJobs	
\end{itemize}
\textbf{Comedy}
\begin{itemize}
\item	Comedy Central Jokes.com	
\item	Kickass jokes	
\end{itemize}
\textbf{Entertainment}
\begin{itemize}
\item	E! Online	
\item	Entertainment Weekly	
\item	TMZ	
\end{itemize}
\textbf{Food}
\begin{itemize}
\item	AllRecipes	
\item	Cooking.com	
\item	Food Network	
\item	Food.com	
\item	Meals.com	
\end{itemize}
\textbf{Games}
\begin{itemize}
\item	Addicting games	
\item	Amorgames	
\item	Crazy monkey games	
\item	GameNode	
\item	Games.com	
\item	Miniclip	
\end{itemize}
\textbf{Health}
\begin{itemize}
\item	Centers for Disease Control and Prevention	
\item	Family Practice notebook	
\item	Health Finder	
\item	HealthCentral	
\item	HealthLine	
\item	Healthlinks.net	
\item	Mayo Clinic	
\item	MedicineNet	
\item	MedlinePlus	
\item	University of Iowa hospitals and clinics	
\item	WebMD	
\end{itemize}
\textbf{Images}
\begin{itemize}
\item	DeviantArt	
\item	Flickr	
\item	Fotolia	
\item	Getty Images	
\item	IconFinder	
\item	NYPL Gallery	
\item	OpenClipArt	
\item	Photobucket	
\item	Picasa	
\item	Picsearch	
\item	Wikimedia	
\item	Funny or Die	
\end{itemize}
\textbf{Kids}
\begin{itemize}
\item	Cartoon Network	
\item	Disney Family	
\item	Factmonster	
\item	Kidrex	
\item	KidsClicks!	
\item	Nick jr	
\item	OER Commons	
\item	Quintura Kids	
\end{itemize}
\textbf{Music}
\begin{itemize}
\item	LastFM	
\item	LYRICSnMUSIC	
\end{itemize}
\textbf{News}
\begin{itemize}
\item	BBC	
\item	Chronicling America	
\item	CNN	
\item	Forbes	
\item	JSOnline	
\item	Slate	
\item	The Street	
\item	Washington post	
\end{itemize}
\textbf{Products}
\begin{itemize}
\item	Amazon	
\item	ASOS	
\item	Craigslist	
\item	eBay	
\item	Overstock	
\item	Powell's	
\item	Pronto	
\item	Target	
\item	Yahoo! Shopping	
\end{itemize}
\textbf{Q\&A}
\begin{itemize}
\item	AllExperts	
\item	Answers.com	
\item	Chacha	
\item	StackOverflow	
\item	Yahoo Answers	
\item	MetaOptimize	
\item	HowStuffWorks	
\end{itemize}
\textbf{Science \& Knowledge}
\begin{itemize}
\item	arXiv.org	
\item	CCSB	
\item	CERN Documents	
\item	CiteSeerX	
\item	CiteULike	
\item	eScholarship	
\item	KFUPM ePrints	
\item	MPRA	
\item	MS Academic	
\item	Nature	
\item	Organic Eprints	
\item	SpringerLink	
\item	U. Twente	
\item	UAB Digital	
\item	UQ eSpace	
\item	PubMed	
\item	Wikipedia	
\item	Wikispecies	
\item	Wiktionary	
\item	National geographic	
\end{itemize}
\textbf{Search}
\begin{itemize}
\item	About.com	
\item	Ask	
\item	CMU ClueWeb	
\item	Gigablast	
\item	Baidu	
\end{itemize}
\textbf{Social}
\begin{itemize}
\item	Foursquare	
\item	Myspace	
\item	Reddit	
\item	Tweepz	
\end{itemize}
\textbf{Sport}
\begin{itemize}
\item	bleacher report	
\item	ESPN	
\item	Fox Sports	
\item	NHL	
\item	SB nation	
\item	Sporting news	
\item	WWE	
\end{itemize}
\textbf{Technology}
\begin{itemize}
\item	HNSearch	
\item	Slashdot	
\item	The Register	
\item	4Shared	
\item	Cnet	
\item	GitHub	
\item	SourceForge	
\item	Ars Technica	
\item	CNET	
\item	Technet	
\item	Technorati	
\item	TechRepublic	
\end{itemize}
\textbf{Travel}
\begin{itemize}
\item	TripAdvisor	
\item	Wiki Travel	
\end{itemize}
\textbf{Video}
\begin{itemize}
\item	Comedy Central	
\item	Dailymotion	
\item	YouTube	
\item	IMDb	
\item	5min.com	
\item	AOL Video	
\item	Google Videos	
\item	MeFeedia	
\item	Metacafe	
\item	Veoh	
\item	Vimeo	
\item	Yahoo Screen	
\item	BigWeb	
\end{itemize}


\end{document}
